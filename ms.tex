\documentclass[twocolumn]{aastex63}
\usepackage{apjfonts}

\usepackage{lipsum}  

%\received{\today}
%\revised{\today}
%\accepted{\today}
%% Command to document which AAS Journal the manuscript was submitted to.
%% Adds "Submitted to " the argument.
%\submitjournal{AJ}

%\topmargin 0.6in

\usepackage{multirow}

%\usepackage{deluxetable}
%\usepackage{color}
%\usepackage[usenames, dvipsnames]{color}
%\definecolor{citeRGB}{rgb}{0,0.1,0.7}
%\usepackage[hyperfootnotes=true,naturalnames=true,letterpaper,pdfstartview=FitH,pdfpagemode=UseNone,colorlinks=true,citecolor=citeRGB]{hyperref}

\gdef\HST{\textit{HST}}
\gdef\Spitzer{\textit{Spitzer}}
\gdef\fluxcgs{\mathrm{erg~s^{-1}~cm^{-2}}}
\gdef\micront{$\mu$m}
\gdef\micronm{\mu\mathrm{m}}
\gdef\aXe{\texttt{aXe}}
\gdef\flux_radius{\textsc{flux\_radius}}
\gdef\epers{\textit{e}$^{-}$ s$^{-1}$}

\gdef\logOH{12 + \log \left(\mathrm{O/H}\right)}
\gdef\SFRuvir{SFR_\mathrm{UV+IR}}
\gdef\peryr{\mathrm{yr}^{-1}}
\gdef\kms{km\,s$^{-1}$}
\gdef\mum{$\mu\mathrm{m}$}
\gdef\24mum{$24\,\mu\mathrm{m}$}
\gdef\arcsec{^{\prime\prime}}
\gdef\UDFj{UDFj-39546284}
\gdef\compareID{UDF-40106456}
\gdef\Lya{\mathrm{Ly}\alpha}
\gdef\Halpha{\mathrm{H}\alpha}
\gdef\Hbeta{\mathrm{H}\beta}

\newcommand\xxx{{\textcolor{red}{\bf xxx}}}
\newcommand\xref[1]{{\textcolor{red}{\bf (REF #1)}}}
\newcommand\XXX[1]{{\textcolor{red}{xxx #1}}}

\gdef\HAWKI{\mbox{HAWK-I}}

\shortauthors{Brammer et al.}
\shorttitle{CHArGE}
% \setwatermarkfontsize{50pt} 
%\citestyle{aa}
\begin{document}

\title{CHArGE: The Complete Hubble Archive for Galaxy Evolution}

% \footnotetext[*]{Based on observations made with the NASA/ESA \textit{Hubble
% Space Telescope}, programs GO-11640, 12177 and 12328, obtained at the
% Space Telescope Science Institute, which is operated by the Association of
% Universities for Research in Astronomy, Inc., under NASA contract NAS
% 5-26555.}

\correspondingauthor{Gabriel Brammer}
\email{gabriel.brammer@nbi.ku.dk}

\author[0000-0003-2680-005X]{Gabriel Brammer}
\affiliation{The Cosmic Dawn Center, University of Copenhagen, Vibenshuset, Lyngbyvej 2, DK-2100 Copenhagen, Denmark}

%%%%%%%%%%%%%%%%%%%%%%%%%%%%%%%%%%%%%%%
%%%%%%  Abstract
%%%%%%%%%%%%%%%%%%%%%%%%%%%%%%%%%%%%%%%
\begin{abstract}

With its high spatial resolution ($\approx 0\farcs1\,\mathrm{pix}^{-1}$) and relatively small imaging detectors, the \textit{Hubble Space Telescope} is not generally considered as a wide-area survey facility.

\end{abstract}
\keywords{galaxies: evolution --- galaxies: high-redshift --- surveys}

%%%%%%%%%%%%%%%%%%%%%%%%%%%%%%%%%%%%%%%
%%%%%% Introduction
%%%%%%%%%%%%%%%%%%%%%%%%%%%%%%%%%%%%%%%
\section{Introduction}
\label{s:introduction}

\begin{itemize}
    \item Imaging surveys
    \item Grism Surveys
    \item high-redshift selection
\end{itemize}

%\lipsum[1-4]

%%%%%%%%%%%%%%%%%%%%%%%%%%%%%%%%%%%%%%%
%%%%%% Archival data summary
%%%%%%%%%%%%%%%%%%%%%%%%%%%%%%%%%%%%%%%
\section{Observations}
\label{s:observations}

%%%%%%%%%%%%%%%%%%%%%%%%%%%%%%%%%%%%%%%
%%%%%%    --- Associations
%%%%%%%%%%%%%%%%%%%%%%%%%%%%%%%%%%%%%%%
\subsection{Data associations}
\label{s:associations}

\begin{itemize}
    \item Automatic associations
    \url{https://github.com/gbrammer/mastquery}
    \item By instrument / filter / visit+epoch / dispersion
\end{itemize}

\subsection{Survey summary}
\label{s:survey}

\begin{itemize}
    \item Depth
    \item Area
    \item Filter coverage
\end{itemize}

\begin{deluxetable*}{lr}
\tablenum{1}
\tablecaption{Coverage area by filter}
\tablewidth{0pt}
\tablehead{
    \colhead{HST Filter} & \colhead{ Area (deg$^2$)}
}

\startdata
F098M & 0.2 \\
F105W & 0.8 \\
F110W & 1.1 \\
F125W & 1.0 \\
F140W & 1.2 \\
F160W & 2.5 \\
F350LP & 0.4 \\
F435W & 0.2 \\
F555W & 0.1 \\
F600LP & 0.1 \\
F606W & 2.0 \\
F625W & 0.2 \\
F775W & 0.6 \\
F814W & 2.5 \\
F850LP & 0.7 \\
\hline
G102 & 0.5 \\
G141 & 0.9 \\
G800L & 0.6 \\
\hline
$\mathrm{opt} = i_{775} | i_{814} | z_{850}$   & 3.4 \\
$Y = y_{098 } | y_{105}$                       & 1.9 \\
$J = j_{110} | j_{125}$                        & 1.0 \\
$H = h_{140} | h_{160}$                        & 3.1 \\
$\mathrm{opty} = \mathrm{opt} | Y$             & 3.8 \\
$Y-\mathrm{drop} = (\mathrm{opt} | Y) + J + H$ & 1.1 \\
$J-\mathrm{drop} = (Y | J) + H$                & 1.7 \\
\enddata
\end{deluxetable*}

%%%%%%%%%%%%%%%%%%%%%%%%%%%%%%%%%%%%%%%
%%%%%%    --- External data
%%%%%%%%%%%%%%%%%%%%%%%%%%%%%%%%%%%%%%%
\subsection{External Data}
\label{s:external}

Generally useful archival data

\begin{itemize}
    \item Spitzer 
    \item VLT: MUSE, HAWKI
    \item Large ground surveys like HSC, VHS, VIDEO?
    \item ALMA?
\end{itemize}

%%%%%%%%%%%%%%%%%%%%%%%%%%%%%%%%%%%%%%%
%%%%%%  Data processing
%%%%%%%%%%%%%%%%%%%%%%%%%%%%%%%%%%%%%%%
\section{Data processing}
\label{s:procesing}

%%%%%%%%%%%%%%%%%%%%%%%%%%%%%%%%%%%%%%%
%%%%%%    --- Preprocessing
%%%%%%%%%%%%%%%%%%%%%%%%%%%%%%%%%%%%%%%
\subsection{Preprocessing}
\label{s:preprocessing}

Things common to both slitless and imaging: 
\begin{itemize}
    \item Background
    \item Satellite trails
    \item Asteroids (not automatic)
    \item Data products (flc, flt)
\end{itemize}

%%%%%%%%%%%%%%%%%%%%%%%%%%%%%%%%%%%%%%%
%%%%%%    --- Imaging
%%%%%%%%%%%%%%%%%%%%%%%%%%%%%%%%%%%%%%%
\subsection{Imaging}
\label{s:process_imaging}

%%%%%%%%%%%%%%%%%%%%%%%%%%%%%%%%%%%%%%%
%%%%%%    ---.--- Imaging.Astrometry
%%%%%%%%%%%%%%%%%%%%%%%%%%%%%%%%%%%%%%%
\subsubsection{Astrometry}
\label{s:astrometry}

To make full use of the multi-wavelength \HST\ datasets, all images in a given filter and across filters/grisms must be aligned to a relative precision somewhat better than the instrumental pixel scales ($\lesssim 0\farcs05 = 50\,mas$).  Requirements on the absolute astrometric precision depends on the degree to which the \HST\ observations are used in consort with observations from other telescopes, such as imaging from the ground.  With relatively poor image quality from the ground the required precision is not particularly high; excellent $0\farcs5$ seeing corresponds to 5--10 \HST\ pixels.  However, the Atacama Large Millimeter Array (ALMA) can now achieve image quality better than \xxx\ mas \xref{} and with absolute astrometric precision \xxx\ mas, and the spatial comparison between stellar emission traced by \HST\ in the optical/NIR and gas and dust traced by ALMA is of paramount scientific interest in the recent literature \xref{Franco, Carlos, etc.}. Since only the target of interest (if that) is generally visible in a given ALMA map, relative alignment between \HST\ and ALMA is impossible and therefore places strong requirements on the \HST\ absolute astrometry.

The absolute \textit{a priori} astrometric precision of \HST\ observations as downloaded from the MAST archive is limited by the astrometric uncertainty of guide stars in the Guide Star Catalog 2 \xref{}, which is generally \xxx\ mas and can be substantially larger for guide stars with significant but uncertain proper motions.  Repeat \HST\ observations of a given field will not necessarily even use the same guide stars, which can result in significant misalignment in both the relative and absolute sense.  The relative astrometric precision for (e.g., dithered) exposures taken within an \HST\ visit of less than a few orbits is generally \xxx\ mas.  However, intra-visit drifts of \xxx\ mas or more can occur that are not reflected in the image headers, depending on the functionality of the \HST\ gyroscopes that control and maintain pointing of the spacecraft \xref{}.  

\XXX{Figure showing astrometric accuracy, from exposure\_log.product.cat.fits catalogs gaia, etc}

%%%%%%%%%%%%%%%%%%%%%%%%%%%%%%%%%%%%%%%
%%%%%%    ---.--- Imaging.Mosaics
%%%%%%%%%%%%%%%%%%%%%%%%%%%%%%%%%%%%%%%
\subsubsection{Mosaics}
\label{s:mosaics}

%%%%%%%%%%%%%%%%%%%%%%%%%%%%%%%%%%%%%%%
%%%%%%    ---.--- Imaging.Photometry
%%%%%%%%%%%%%%%%%%%%%%%%%%%%%%%%%%%%%%%
\subsubsection{Photometry}
\label{s:photometry}

%%%%%%%%%%%%%%%%%%%%%%%%%%%%%%%%%%%%%%%
%%%%%%    --- WFSS
%%%%%%%%%%%%%%%%%%%%%%%%%%%%%%%%%%%%%%%
\subsection{Slitless Spectroscopy}
\label{s:process_slitless}

%%%%%%%%%%%%%%%%%%%%%%%%%%%%%%%%%%%%%%%
%%%%%%    ---.--- Slitless.Calibration
%%%%%%%%%%%%%%%%%%%%%%%%%%%%%%%%%%%%%%%
\subsubsection{Calibration}
\label{s:slitless_calibration}

%%%%%%%%%%%%%%%%%%%%%%%%%%%%%%%%%%%%%%%
%%%%%%    ---.--- Slitless.Astrometry
%%%%%%%%%%%%%%%%%%%%%%%%%%%%%%%%%%%%%%%
\subsubsection{Astrometry}
\label{s:slitless_astrometry}

%%%%%%%%%%%%%%%%%%%%%%%%%%%%%%%%%%%%%%%
%%%%%%    ---.--- Slitless.Contamination
%%%%%%%%%%%%%%%%%%%%%%%%%%%%%%%%%%%%%%%
\subsubsection{Contamination}
\label{s:contamination}

%%%%%%%%%%%%%%%%%%%%%%%%%%%%%%%%%%%%%%%
%%%%%%    ---.--- Slitless.Extraction
%%%%%%%%%%%%%%%%%%%%%%%%%%%%%%%%%%%%%%%
\subsubsection{Spectral extraction}
\label{s:extraction}

%%%%%%%%%%%%%%%%%%%%%%%%%%%%%%%%%%%%%%%
%%%%%%    ---.--- Slitless.Redshift 
%%%%%%%%%%%%%%%%%%%%%%%%%%%%%%%%%%%%%%%
\subsubsection{Redshift Determination}
\label{s:redshift}

%%%%%%%%%%%%%%%%%%%%%%%%%%%%%%%%%%%%%%%
%%%%%%    ---.--- Slitless.Line Maps 
%%%%%%%%%%%%%%%%%%%%%%%%%%%%%%%%%%%%%%%
\subsubsection{Emission Line Maps}
\label{s:linemaps}

%%%%%%%%%%%%%%%%%%%%%%%%%%%%%%%%%%%%%%%
%%%%%%  Database
%%%%%%%%%%%%%%%%%%%%%%%%%%%%%%%%%%%%%%%
\subsection{\textit{Spitzer} Imaging}
\label{s:spitzer}

%%%%%%%%%%%%%%%%%%%%%%%%%%%%%%%%%%%%%%%
%%%%%%  Database
%%%%%%%%%%%%%%%%%%%%%%%%%%%%%%%%%%%%%%%
\section{Measurement Database}
\label{s:database}

%%%%%%%%%%%%%%%%%%%%%%%%%%%%%%%%%%%%%%%
%%%%%%    --- Discussion
%%%%%%%%%%%%%%%%%%%%%%%%%%%%%%%%%%%%%%%
\section{Discussion}
\label{s:discussion}

%%%%%%%%%%%%%%%%%%%%%%%%%%%%%%%%%%%%%%%
%%%%%%    --- Cloud
%%%%%%%%%%%%%%%%%%%%%%%%%%%%%%%%%%%%%%%
\subsection{Data Processing in the Cloud}
\label{s:cloud}

%%%%%%%%%%%%%%%%%%%%%%%%%%%%%%%%%%%%%%%
%%%%%%  Future missions
%%%%%%%%%%%%%%%%%%%%%%%%%%%%%%%%%%%%%%%
\subsection{Future Missions}
\label{s:future}

\begin{itemize}
    \item NIRISS
    \item Euclid
    \item WFIRST
\end{itemize}


%%%%%%%%%%%%%%%%%%%%%%%%%%%%%%%%%%%%%%%
%%%%%%  End things
%%%%%%%%%%%%%%%%%%%%%%%%%%%%%%%%%%%%%%%
\acknowledgments

Software acknowledgements: astropy, scipy, matplotlib, sep, astroquery, sqlalchemy, shapely

\\ The Cosmic Dawn Center (DAWN) is funded by the Danish National Research Foundation under grant No. 140. 

\end{document}
